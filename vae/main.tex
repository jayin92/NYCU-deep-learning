\documentclass[a4paper]{article}
\usepackage{fontspec}   %加這個就可以設定字體
\usepackage{xeCJK}       %讓中英文字體分開設置
\usepackage{indentfirst}
\usepackage{listings}
\usepackage[newfloat]{minted}
\usepackage{float}
\usepackage{graphicx}
\usepackage{caption}
\usepackage{fancyhdr}
\usepackage{hyperref}
\usepackage{amsmath}
\usepackage{multirow}
\usepackage[dvipsnames]{xcolor}
\usepackage{graphicx}
\usepackage{tabularx}
\usepackage{booktabs}
\usepackage{caption}
\usepackage{subcaption}
\usepackage{pifont}
\usepackage{amssymb}

\usepackage{pdftexcmds}
\usepackage{catchfile}
\usepackage{ifluatex}
\usepackage{ifplatform}

\usepackage[breakable, listings, skins, minted]{tcolorbox}
\usepackage{etoolbox}
\setminted{fontsize=\footnotesize}
\renewtcblisting{minted}{%
    listing engine=minted,
    minted language=python,
    listing only,
    breakable,
    enhanced,
    minted options = {
        linenos, 
        breaklines=true, 
        breakbefore=., 
        % fontsize=\footnotesize, 
        numbersep=2mm
    },
    overlay={%
        \begin{tcbclipinterior}
            \fill[gray!25] (frame.south west) rectangle ([xshift=4mm]frame.north west);
        \end{tcbclipinterior}
    }   
}

\usepackage[
top=1.5cm,
bottom=0.75cm,
left=1.5cm,
right=1.5cm,
includehead,includefoot,
heightrounded, % to avoid spurious underfull messages
]{geometry} 

\newenvironment{code}{\captionsetup{type=listing}}{}
\SetupFloatingEnvironment{listing}{name=Code}



\title{Deep Learning Additional Homework - Variational Autoencoder}
\author{110550088 李杰穎}
\date{\today}


\setCJKmainfont{Noto Serif TC}



\ifwindows
\setmonofont[Mapping=tex-text]{Consolas}
\fi

\XeTeXlinebreaklocale "zh"             %這兩行一定要加,中文才能自動換行
\XeTeXlinebreakskip = 0pt plus 1pt     %這兩行一定要加,中文才能自動換行

\newcommand*{\dif}{\mathop{}\!\mathrm{d}}

\begin{document}
\maketitle

\section{Closed-form function of KL term}

Here, we need to derive the closed-form function of KL term, $\text{KL}\left(q(\mathbf{Z}|\mathbf{X};\theta') \, || \, p(\mathbf{Z})\right)$. Suppose $Q \sim \mathcal{N}(\mu, \sigma^2)$ and the prior distribution $p(\mathbf{Z})$ is $\mathcal{N}(0, 1)$, the KL term can be expanded as:
\begin{align}
\text{KL}\left(q(\mathbf{Z}|\mathbf{X};\theta') \, || \, p(\mathbf{Z})\right) &= \int_z q(z|\mathbf{X};\theta') \ln \frac{q(z|\mathbf{X};\theta')}{p(z)} \\
&= \int_{-\infty}^{\infty} \mathcal{N}(z; \mu, \sigma^2) \ln \frac{\mathcal{N}(z; \mu, \sigma^2)}{\mathcal{N}(z; 0, 1)} \, dz \\
&= \mathbb{E}\left[\ln \frac{\mathcal{N}(\mathbf{Z}; \mu, \sigma^2)}{\mathcal{N}(\mathbf{Z}; 0, 1)} \right] \\
&= \mathbb{E}\left[\ln \frac{\frac{1}{\sqrt{2 \pi \sigma^2}} \exp(\frac{-1}{2} \left(\frac{\mathbf{Z}-\mu}{\sigma}\right)^2)}{\frac{1}{\sqrt{2 \pi}} \exp \left(\frac{-1}{2} \mathbf{Z}^2\right)} \right] \\
&= \mathbb{E}\left[\frac{1}{2} \ln \frac{1}{\sigma^2} - \frac{1}{2} \left(\frac{\mathbf{Z}-\mu}{\sigma}\right)^2 + \frac{1}{2} \mathbf{Z}^2 \right] \\
&= \frac{1}{2} \mathbb{E}\left[\ln \frac{1}{\sigma^2} - \left(\frac{\mathbf{Z}-\mu}{\sigma}\right)^2 + \mathbf{Z}^2 \right] \\
&= \frac{1}{2} \left( \ln \frac{1}{\sigma^2} - \frac{\mathbb{E}\left[(\mathbf{Z} - \mu)^2\right]}{\sigma^2} + \mathbb{E}\left[\mathbf{Z}^2 \right] \right).
\end{align}

Because $z \sim \mathcal{N}(\mu, \sigma^2)$, by definition, we have $\mathbb{E}\left[(\mathbf{Z} - \mu)^2\right] = \sigma^2$. Also, $\text{Var}\left[Q\right] = \sigma^2 = \mathbb{E}\left[\mathbf{Z}^2\right] - \left(\mathbb{E}\left[\mathbf{Z}\right]\right)^2$. Thus, we have $\mathbb{E}\left[\mathbf{Z}^2 \right] = \mu^2 + \sigma^2$. Finally, the closed-form function can be written as:
\begin{align}
&= \frac{1}{2} \left( \ln \frac{1}{\sigma^2} - \frac{\mathbb{E}\left[(\mathbf{Z} - \mu)^2\right]}{\sigma^2} + \mathbb{E}\left[\mathbf{Z}^2 \right] \right) \\
&= \frac{1}{2} \left( -\ln \sigma^2 - 1 + \mu^2 + \sigma^2 \right).
\end{align}

\section{Gaussian Mixture Model as Prior}


The Gaussian mixture model (GMM) combines multiple Gaussian distributions by calculating the weighted sum of each Gaussian distributions. Specifically, a GMM can be formally written as: 
\begin{equation}
p(\mathbf{Z}) = \sum_{k=1}^K \pi_k \mathcal{N}(\mathbf{Z}|\boldsymbol{\mu}_k, \boldsymbol{\Sigma}_k)
\end{equation}
where $\pi_k$ are mixture weights (with $\sum_{k=1}^K \pi_k = 1$), and $\boldsymbol{\mu}_k, \boldsymbol{\Sigma}_k$ are the mean and covariance of each component. Compared with standard normal distribution, GMM can capture multi-modal data distributions more effectively, increasing the VAE's representational capacity. However, the issue arises when calculating the closed-form function of the KL term. The KL term doesn't have closed-form solution when using GMM as prior distribution, due to the summation in logarithm. The lack of closed-form solution prevents us not able to efficiently optimize the encoder network through back-propagation. To deal with this, we can use Monte Carlo sampling to estimate the KL divergence. Specifically, we can evaluate KLD as follow:
\begin{equation}
\text{KL}\left(q(\mathbf{Z}|\mathbf{X};\theta') \, || \, p(\mathbf{Z})\right) \approx \frac{1}{N}\sum_{i=1}^{N} \ln \frac{q(z_i|\mathbf{X};\theta')}{p(z_i)}, 
\end{equation}
where $z_i$ samples from $q(\mathbf{Z}|\mathbf{X};\theta')$, $N$ is the number of samples and $p(\mathbf{Z})$ is the prior GMM distribution.



\end{document}